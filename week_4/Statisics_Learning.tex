\documentclass{beamer}
%
% Choose how your presentation looks.
%
% For more themes, color themes and font themes, see:
% http://deic.uab.es/~iblanes/beamer_gallery/index_by_theme.html
%
\mode<presentation>
{
  \usetheme{default}      % or try Darmstadt, Madrid, Warsaw, ...
  \usecolortheme{default} % or try albatross, beaver, crane, ...
  \usefonttheme{default}  % or try serif, structurebold, ...
  \setbeamertemplate{navigation symbols}{}
  \setbeamertemplate{caption}[numbered]
} 

\usepackage[english]{babel}
\usepackage[utf8x]{inputenc}
\usepackage{amsmath,amsthm,amssymb,amsfonts}
\newtheorem*{dfn}{Definition}
\newtheorem{thm}{Theorem}[subsection]
 \renewcommand{\thethm}{\arabic{thm}}
%\newtheorem{lemma}{Lemma}
\title[Introduction]{Statistics Learning Theorem:General PAC model with finite class}
\author{}
\institute{}
\date{2019.02.09}

\begin{document}

\begin{frame}
  \titlepage
\end{frame}

% Uncomment these lines for an automatically generated outline.
%\begin{frame}{Outline}
%  \tableofcontents
%\end{frame}

\section{Review of PAC Model}
\begin{frame}{Agnostic PAC Learnability for General Loss Functions}
	A hypothesis class $\mathcal{H}$ is agnostic PAC learnable with respect to a set $Z$ and a loss function $l: \mathcal{H} \times Z \rightarrow R_+$, if there exist a function $m_{\mathcal{H}}:(0,1)^2 \rightarrow N$ and a learning algorithm with the following preperty: For every $\varepsilon,\delta \in (0,1)$ and for every distribution $\mathcal{D}$ over $Z$, when running the learning algorithm on $m \geq m_{\mathcal{H}} (\varepsilon,\delta)$ iid samples generated by $\mathcal{D}$, the algorithm returns $h \in \mathcal{H}$ such that, with probability of at least $1- \delta$
	\[L_{\mathcal{D}} \leq \min_{h' \in \mathcal{H}} L_{\mathcal{D}} (h') + \varepsilon\]
	where $L_{\mathcal{D}}(h) = E_{z \sim \mathcal{D}} [l(h,z)]$
\end{frame}
\section{General PAC with finite class}
\begin{frame}{Goal}
	In today's slide, we aim to show that finite hypothesis class is agnostic PAC learnable with general learnable under uniform covergence condition. 
\end{frame}
\begin{frame}{Some terminology}
	\begin{dfn}[$\varepsilon$-representative sample]
	A training set $S$ is called $\varepsilon$-representative sample if 
	\[\forall h \in \mathcal{H}, \quad |L_S(h) - L_{\mathcal{D}}(h)| \leq \varepsilon\]
	\end{dfn}
	\begin{dfn}[Uniform Convergence]
		We say that a hypothesis  class $\mathcal{H}$ has the uniform convergence property, if there exists a function $m_{\mathcal{H}}^{UC}:(0,1)^2 \rightarrow N$ such that for every $\varepsilon,\delta \in (0,1)$ and for every probability distribution $\mathcal{D}$ over $Z$, if $S$ is a sample of $m \geq m_{\mathcal{H}}^{UC}(\varepsilon,\delta)$ examples drawn i.i.d, according to $\mathcal{D}$, then, with probability of at least $1-\delta$, $S$ is $\varepsilon$-representative sample
	\end{dfn}

\end{frame}
\begin{frame}{Main Theorem}
	\begin{thm}
Every finite hypothesis class is agnostic PAC learnable
\end{thm}
\end{frame}
\begin{frame}{Proof}
	Before the starting the proof, we set up some lemma in the fist.
	\begin{lemma}
		Assume that a training set $S$ is $\frac{\varepsilon}{2}$-representative sample. Then, any output of $ERM_{\mathcal{H}}(S)$,namely, any $h_S \in arg\min_{h \in \mathcal{H}} L_S(h)$, satisfies
		\[L_{\mathcal{D}}(h_S) \leq \min_{h \in \mathcal{H}} L_{\mathcal{D}} (h) + \varepsilon\]
	\end{lemma}
	The proof is simple, it follows that 
	\[L_{\mathcal{D}} (h_S) \leq L_S(h_S)  + \frac{\varepsilon}{2} \leq L_S(h) + \frac{\varepsilon}{2} \leq L_{\mathcal{D}} (h) + \varepsilon\]
\end{frame}
\begin{frame}{Proof}
	From the above lemma, we know that the ERM rule is agnostic PAC learnable if given $\varepsilon,\delta$ we can alway find $\varepsilon$-representative sample under $1-\delta$ probaility.  \\
	That is, the definition of uniform convergence. Therefore, we can get the following corollary:
	\begin{thm}
		If a class $\mathcal{H}$ has the uniform convergence property with a function $m_{\mathcal{H}}^{UC}$ then the class is agnostically PAC learnable with the sample complexity $m_{\mathcal{H}} (\varepsilon,\delta) \leq m_{\mathcal{H}}^{UC}(\frac{\varepsilon}{2},\delta)$. Furthermore, in that case, the $ERM_{\mathcal{H}}$ paradigm is a sucessful agnostic PAC learner for $\mathcal{H}$
	\end{thm}
\end{frame}
\begin{frame}{Proof:Framework}
	By above theorem, we only need to show that the finite class $\mathcal{H}$ satisfies the property of uniform convergence. \\
	We follow a two step argument. The first step applies the union bound and the second step employs a measure concentration inequality.
\end{frame}
\begin{frame}{Proof:First Step}
	Fix some $\varepsilon,\delta$, we want to show that 
	\[D^{m}(\{S: \exists h \in \mathcal{H}, |L_S(h) - L_D(h)| > \varepsilon\}) < \delta\]
	We know that
	\[\{S: \exists h \in \mathcal{H}, |L_S(h) - L_D(h)| > \varepsilon\} =\cup_{h \in \mathcal{H}} \{S: |L_S(h) - L_D(h)| > \varepsilon\}\]
	Therefore,
	\[D^{m}(\{S: \exists h \in \mathcal{H}, |L_S(h) - L_D(h)| > \varepsilon\})\]
	\[\leq \sum_{h \in \mathcal{H}} D^m(\{S:|L_S(h) - L_D(h)| > \varepsilon\})\]
\end{frame}
\begin{frame}{Proof:Second Step}
	Since $L_D(h) = E[l(h,z)]$, $L_S = \frac{1}{m} \sum^m_{i=1} l(h,z_i)$, the iid assumption and law of large number make sure the each summand would converge to zero. However, we want to estimate the sample complexity, it requires to estimate the upper bound of concentrated error under finite error. To achieve it, we use the Hoeffding's Inequality:
	\begin{lemma}[Hoeffding's Inequality]
		Let $\theta_1, \cdots , \theta_m$ be a sequence of i.i.d. random variables and assume that for all $i, E[\theta_i] = \mu$ and $P[a\leq \theta_i \leq b]=1$. Then, for any $\varepsilon >0$
		\[P[|\frac{1}{m} \sum^m_{i=1} \theta_i - \mu| > \varepsilon] \leq 2 exp(-2 m \varepsilon^2/(b-a)^2)\]
	\end{lemma}
\end{frame}
\begin{frame}{Proof:Second Step}
	Using the same notation in the lemma($L_S(h) = \frac{1}{m} \sum^m_{i=1} \theta_i,L_D(h) = \mu$), we have that 
	\[D^{m}(\{S: |L_S(h) - L_D(h)| > \varepsilon\})\]
\[=P[|\frac{1}{m} \sum^m_{i=1} \theta_i - \mu| > \varepsilon] \leq 2 exp(-2 m \varepsilon^2)\]
Therefore, we have
\[D^{m}(\{S: \exists h \in \mathcal{H}, |L_S(h) - L_D(h)| > \varepsilon\})\]
\[\leq \sum_{h \in \mathcal{H}} 2 exp(-2m \varepsilon^2)\]
\[= 2 |\mathcal{H}| exp(-2m\varepsilon^2)\]
That is , we choose 
\[m  \geq \frac{\log(2|\mathcal{H}|/\delta)}{2\varepsilon^2}\]
\end{frame}
\end{document}
