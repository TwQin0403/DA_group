\documentclass{beamer}
%
% Choose how your presentation looks.
%
% For more themes, color themes and font themes, see:
% http://deic.uab.es/~iblanes/beamer_gallery/index_by_theme.html
%
\mode<presentation>
{
  \usetheme{default}      % or try Darmstadt, Madrid, Warsaw, ...
  \usecolortheme{default} % or try albatross, beaver, crane, ...
  \usefonttheme{default}  % or try serif, structurebold, ...
  \setbeamertemplate{navigation symbols}{}
  \setbeamertemplate{caption}[numbered]
} 

\usepackage[english]{babel}
\usepackage[utf8x]{inputenc}

\title[Introduction]{Statistics Learning Theorem:PAC Learing}
\author{}
\institute{}
\date{2019.02.02}

\begin{document}

\begin{frame}
  \titlepage
\end{frame}

% Uncomment these lines for an automatically generated outline.
%\begin{frame}{Outline}
%  \tableofcontents
%\end{frame}

\section{Supplement}

\begin{frame}{Review:Finite Hypothesis Classes- Assumptions}
Assume that 
\begin{itemize}
	\item Realizability Assumption:There exist $h^* \in \mathcal{H}$ such that $L_{D,f}(h^*)=0$
	\item i.i.d assumption: we assume that the training data $S = \{x_1,x_2, \cdots,x_m\}$ are sampled i,i,d from $\mathcal{D}$. Therefore, the distribution of $S$ is $D^m$
	\item $ |\mathcal{H}|$ is finite
\end{itemize}

\end{frame}

\begin{frame}{Review:Finite Hypothesis Classes - Theorem}
	Under above three assumptions, let $\delta \in (0,1)$ and $\varepsilon>0$ and let $m$ be an interger that satisfies 
	\[m \geq \frac{\log (|\mathcal{H}|/ \delta)}{\varepsilon}\]
	Then, for any labeling function $f$, and for any distribution, $\mathcal{D}$, with probability of at least $1-\delta$ over the choice of an sample $S$ of size $m$, we have every ERM prediction rule $h_S$, it holds that 
	\[L_{D,f} (h_S) \leq \varepsilon\]
	
\end{frame}
\begin{frame}{Review:Finite Hypothesis Classes -Proof}
	We try to uppper bound
	\[\mathcal{D}^m (\{S: L_{D,f} (h_S)> \varepsilon\})\]
	Let 
	\[\mathcal{H}_B = \{h \in \mathcal{H}: L_{(D,f)}(h) > \varepsilon\}\]
	\[M = \{S: \exists h \in \mathcal{H}_B,L_S(h)=0\}\]
	Where $\mathcal{H}_B$ collect those 'bad' prediction rule and $M$ collect those sample to mislead the learner.
\end{frame}
\begin{frame}{Review:Finite Hypothesis Classes--Proof}
	Then we have 
	\[\mathcal{D}^m(\{S: L_{D,f} (h_S)> \varepsilon\}) \leq \mathcal{D}^m(M) \leq \sum_{h \in \mathcal{H}_B} \mathcal{D}^m(\{S: L_S(h)=0\})\]
	and 
	\[\mathcal{D}^m(\{S: L_S(h)=0\})= \mathcal{D}^m(\{S: \forall i, h(x_i) = f(x_i))\]
		\[=\prod^m_{i=1} \mathcal{D}(\{x_i: h(x_i) = f(x_i)\})\]
	and we know that $\forall h \in \mathcal{H}_B$
	\[\mathcal{D}(\{x_i: h(x_i) = f(x_i)\}) = 1 - L_{D,f} (h) \leq 1 - \varepsilon\]
\end{frame}
\begin{frame}{Review:Finite Hypothesis Classes--Proof}
	Since $1-\varepsilon \leq e^{-\varepsilon}$, we have 
	\[\mathcal{D}^m(\{S:L_S(h)=0\}) \leq (1- \varepsilon)^m \leq e^{-\varepsilon m}\]
	Therefore,
	\[\mathcal{D}^m (\{S: L_{D,f} (h_S)> \varepsilon\}) \leq |\mathcal{H}_B e^{-\varepsilon m} \leq |H| e^{-\varepsilon m}\]
	Therefore, given $(\delta,\varepsilon)$, we can bound the above set when sample larege than
	\[\frac{\log(|\mathcal{H}|/\delta)}{\varepsilon}\]
\end{frame}


\begin{frame}{Solution for problem}
	Let $x_i$ be the Bernoulli variable with parameter $p=Pr_{x \sim \mathcal{D}} \{x| h(x) \neq f(x)\}$. That is, 
	\[x_i = \begin{cases} 
			1 \quad p \\
			0 \quad 1-p
	\end{cases}\]
	Then $L_S(h) = \frac{\sum^m_{i=1} x_i}{m}$. The iid assumption implies that 
	\[E_{S \sim \mathcal{D}^m} [L_S(h)] = E_{S \sim \mathcal{D}^m} [\frac{\sum^m_{i=1} x_i}{m}] = \frac{\sum^m_{i=1} E_{x_i}[x_i]}{m} = E[x_i] = p = L_{\mathcal{D},f} (h)\]
	The $E_{S \sim \mathcal{D}^m} [\frac{\sum^m_{i=1} x_i}{m}]= \frac{\sum^m_{i=1} E_{x_i}[x_i]}{m}$ follows from Fubini's theorem.
\end{frame}

\section{PAC Model}
\begin{frame}{Definition of PAC Learnability}
	A hypothesis class $\mathcal{H}$ is PAC learnable if there exist a function $m_{\mathcal{H}}: (0,1)^2 \rightarrow N$ and a learning algorithm with the following property: For every $\varepsilon, \delta \in (0,1)$, for every distribution $\mathcal{D}$ over $\mathcal{X}$, and for every labeling function $f: \mathcal{X} \rightarrow \{0,1\}$, if the realizable assumption holds with respect to $\mathcal{H},\mathcal{D},f,$then when running the learning algorithm returns a hypothesis $h$ such that, with probability of at least $1-\delta$, $L_{\mathcal{D},f}(h) \leq \varepsilon$. \\
\end{frame}
\begin{frame}{Remark}
	\begin{itemize}
	\item Accuracy parameter $\varepsilon$ determines how correct the output classifier can be from the optimal one.
	\item Confidence parameter $\delta$ indicates how likely the classifier meet that accuracy requirement.
	\item Sample Complexity: $m_{\mathcal{H}}$ is the minimal sample to guarantee a probably approximately correct solution.
	\item As a result, finite hypothesis class is PAC with realizable assumption is PAC learnable with sample complexity 
		\[m_{\mathcal{H}}(\varepsilon,\delta) \leq \frac{\log(|\mathcal{H}|/\delta)}{\varepsilon}\]
	\end{itemize}
\end{frame}
\begin{frame}{Example}
	Let $\mathcal{X}$ be a discrete domain, and let $\mathcal{H}=\{h_z:z \in \mathcal{X}\} \cup \{h^{-}\}$, where for each $z \in \mathcal{X},h_z$ is the function defined by $h_z(x)=1$ if $x=z$ and $h_z(x)=0$ if $x \neq z. h^{-}$ is simply the all-negative hypothesis, namely, $\forall x \in X,h^{-}(x)=0$. The realizability assumption here implies that the true hypothesis $f$ labels negatively all examples in the domain, perhaps except one. Then $\mathcal{H}$ is PAC learnable.
\end{frame}
\begin{frame}{Proof of example}
	First, we propose the algorithm. If $L_S(h^{-})=0$, we  choose $h_S = h^{-}$,otherwise, we choose $h_S = h_z$ for some $h_z$ such that $L_S(h_z)=0$. \\
	Given $(\varepsilon,\delta) \in (0,1)^2$, we try to find the corresponding $m_{\mathcal{H}}$. \\ 
	Case 1: $h_S = h_z$, then by realizability assumption, we are done. \\
\end{frame}
\begin{frame}{Proof of example}
	Case 2:$h_S = h^{-}$, assume the label function $f \sim h_{z^*}$, let $p = D\{x = z^*\}$, then we have that 
	\[L_{(D,f)} (h_S) \leq L_{D,f} (h_{z^*}) \leq D^{m}\{S: z^* \notin S\} = (1-p)^m\]
	WLOG we assume $p > \varepsilon$: we hope that $(1-p)^m \leq \delta$, since $1-p \leq 1- \varepsilon$, then we can choose 
	\[m \geq \frac{\log \delta}{\log(1-\varepsilon)}\]
\end{frame}
\begin{frame}{Two way to relax the definition}
	\begin{itemize}
			\item Removing the realizability assumption: Agnostic PAC
			\item Learning Problems beyond Binary Classification: Multi-Classification, regression $\rightarrow$ generalized loss function.
	\end{itemize}

\end{frame}
\begin{frame}{Agnostic PAC Learning}
	There are two aspect we should be able to relax
	\begin{itemize}
		\item Realizability assumption: This assumption assume there exists a almost correct function $h^* \in \mathcal{H}$. This is somehow very strong. 
		\item Labels are fully determined by features: In the original setting, the input features can fully determines the label through $f$. It is not realistic in many practical problems.
	\end{itemize}
	The way out: more realistic model for the data-generating distribution. 

\end{frame}
\begin{frame}{Agnostic PAC Learning}
	To solve the above problems, we allow the distribution $\mathcal{D}$ is over $\mathcal{X} \times \mathcal{Y} $ instead of over $\mathcal{X}$. \\
	In this case, we avoid to introduce the correct labeling function $f$. Instead, we define the true risk as 
	\[L_{\mathcal{D}}(h) = P_{(x,y)} [h(x) \neq  y] = \mathcal{D} (\{(x,y):h(x) \neq y\})\]
	and the empirical risk remains the same as before
\[L_S(h) = \frac{|\{i \in [m]:h(x_i) \neq y_i\}|}{m}\]
\end{frame}
\begin{frame}{Agostic PAC Learning}
	Our goal is to find the some hypothesis,$h: \mathcal{X} \rightarrow \mathcal{Y}$ that minimizes the true risk,$L_D(h)$ \\
	The Bayes Optimal Predictor:
	\[f_{\mathcal{D}} (x) = \begin{cases}
			1 \quad if \ \ P[y=1|x] \geq \frac{1}{2} \\
			0 \quad otherwise
		\end{cases}
		\]
	This solution is optimal in the sense that no other classifier $g$ can have lower risk. That is 
	\[L_{\mathcal{D}} (f_{\mathcal{D}}) \leq L_{\mathcal{D}} (g)\]
\end{frame}
\begin{frame}{Agnostic PAC Learning}
	Before giving the definition of agnostic PAC learning, we give two remarks:
	\begin{itemize}
		\item To avoid putting the realizability assumtption, we cannot expect the learning algorithm can achieve the minimal possible error, that of the Bayes predictor.
		\item We will prove later, once we make no prior assumptions about the data generate process, no algorithm can be guaranteed to find a predictor as good as the Bayes oprimal one.(No Free Lunch Theorem)
	\end{itemize}
	To sum up, we require that the best possible error of a predictor in some given Benchmark hypothesis class.
\end{frame}
\begin{frame}{Agonostic PAC Learning: Definition}
	A hypothesis class $\mathcal{H}$ is agnostic PAC learnable if there exist a function $m_{\mathcal{H}}:(0,1)^2 \rightarrow N$ and a learning algorithm with the following property: For every $\varepsilon, \delta \in (0,1)$ and for every distribution $\mathcal{D}$ over $\mathcal{X} \times \mathcal{Y}$, when running the learning algorithm on $m \geq m_{\mathcal{H}}(\varepsilon,\delta)$ i.i.d. samples generated by $\mathcal{D}$, the algorithm returns a hypothesis $h$ such that, with probability of at least $1-\delta$
	\[L_{\mathcal{D}}(h) \leq \min_{h' \in \mathcal{H}} L_{\mathcal{D}}(h') + \varepsilon\]
\end{frame}
\begin{frame}{Generalized Loss Functions}
There are two important way to extend our PAC model
\begin{itemize}
	\item Multiclass Classification:Document classification(news,sports,biology,medicine): way out:the same structure as before with multiple labels
	\item Regression: House price, way out: need to modify the loss function
\end{itemize}
\end{frame}
\begin{frame}{Regression}
	In the linear regression problem, we general use the following loss function to compute
	\[L_{\mathcal{D}}(h) = E_{(x,y) \sim \mathcal{D}} (h(x) - y)^2\]
	However, the learning framework should be able to accommendate more type of measure of success.
\end{frame}
\begin{frame}{Generalized Loss Functions}
	Given the class $\mathcal{H}$, we can define the generalized loss function as follows:
	\[l:\mathcal{H} \times \mathcal{X} \times \mathcal{Y} \rightarrow R_+\]
	However, if our learning task beyond prediction tasks, we can put $l: \mathcal{H} \times Z$ where $Z$ can be any domain of example.(for example, unsupervised learning)
\end{frame}
\begin{frame}{Generalized Loss Functions}
	we now define the true risk and empirical risk as follows:
	\[L_{\mathcal{D}} (h) = E_{z \sim \mathcal{D}}[l(h,z)]\]
	\[L_S (h) = \frac{1}{m} \sum^m_{i=1} l(h,z_i)\]
\end{frame}
\begin{frame}{Generalized Loss Function:examples}
	\begin{itemize}
			\item 0-1 loss:
				\[l_{0-1} (h,(x,y)) = \begin{cases}
						0 \quad if \ \ h(x) = y	 \\
						1 \quad otherwise
				\end{cases}\]
			\item square loss:
				\[l_{sq} (h,(x,y)) = (h(x) - y)^2\]
	\end{itemize}
\end{frame}
\begin{frame}{Agnostic PAC Learnability for General Loss Functions}
	A hypothesis class $\mathcal{H}$ is agnostic PAC learnable with respect to a set $Z$ and a loss function $l: \mathcal{H} \times Z \rightarrow R_+$, if there exist a function $m_{\mathcal{H}}:(0,1)^2 \rightarrow N$ and a learning algorithm with the following preperty: For every $\varepsilon,\delta \in (0,1)$ and for every distribution $\mathcal{D}$ over $Z$, when running the learning algorithm on $m \geq m_{\mathcal{H}} (\varepsilon,\delta)$ iid samples generated by $\mathcal{D}$, the algorithm returns $h \in \mathcal{H}$ such that, with probability of at least $1- \delta$
	\[L_{\mathcal{D}} \leq \min_{h' \in \mathcal{H}} L_{\mathcal{D}} (h') + \varepsilon\]
	where $L_{\mathcal{D}}(h) = E_{z \sim \mathcal{D}} [l(h,z)]$
\end{frame}
\end{document}
